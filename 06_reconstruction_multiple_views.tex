\section{Reconstruction from Multiple Views}%
\label{sec:reconstruction_from_multiple_views}


\subsection{From Two Views to Multiple Views}%
\label{sub:from_two_views_to_multiple_views}


\subsubsection*{Multiple View Geometry}%
\label{ssub:multiple_view_geometry}


In this section, we deal with the problem of 3D reconstruction
given \textbf{multiple} views of a static scene, either obtained simultaneously,
or sequentially from a moving camera.\\

The key idea is that the three-view scenario allows to obtain more measurements
to infer the same number of 3D coordinates.
For example, given two views of a single 3D point, we have four measurements
(x- and y-coordinate in each view), while the three-view case provides
6 measurements per point correspondence.
As a consequence, the estimation of motion and structure will generally
be more constrained when reverting to additional views.\\

The three-view case has traditionally been addressed by the so-called
\textbf{trifocal tensor} [Hartley '95, Vieville '93]
which generalizes the fundamental matrix.
This tensor --- as the fundamental matrix --- does not depend on the scene
structure but rather on the inter-frame camera motion.
It captures a \textbf{trilinear relationship} between three views of
the same 3D point or line [Liu, Huang '86, Spetsakis, Aloimonos '87].


\subsubsection*{Trifocal Tensor versus Multiview Matrices}%
\label{ssub:trifocal_tensor_versus_multiview_matrices}


Traditionally, the trilinear relations were captured by generalizing
the concept of the Fundamental Matrix to that of a Trifocal Tensor.
It was developed among others by [Liu and Huang '86], [Spetsakis, Aloimonos '87].
The use of tensors was promoted by [Vieville '93] and [Hartley '95].
Bilinear, trilinear and quadrilinear constraints were formulated
in [Triggs '95]. This line of work is summarized in the books:
\textbf{Faugeras and Luong, ``The Geometry of Multiple Views'', 2001} and
\textbf{Hartley and Zisserman, ``Multiple View Geometry'', 2001, 2003}.\\

In the following, however, we stick with a matrix notation for the multiview scenario.
This approach makes use of matrices and rank constraints on these matrices
to impose the constraints from multiple views.
Such rank constraints were used by many authors, among othes in [Triggs '95]
and in [Heyden, Astr\"om '97].
This line of work is summarized in the book
\textbf{Ma, Soatto, Kosecka, Sastry, ``An Invitation to 3D Vision'', 2004}.


\subsection{Preimage \& Coimage from Multiple Views}%
\label{sub:preimage_coimage_from_multiple_views}


\subsubsection*{Preimage from Multiple Views}%
\label{ssub:preimage_from_multiple_views}


A \textbf{preimage of multiple images} of a point or a line is the
(largest) set of 3D points that gives rise to the same set of
multiple images of the point or the line.\\

For example, given the two images $l_1$ and $l_2$ of a line $L$,
the preimage of these two images is the intersection of the planes
$P_1$ and $P_2$, i.e.\ exactly the 3D line $L = P_1 \cap P_2$.\\

In general, the preimage of multiple images of points and lines
can be \textbf{defined by the intersection}:
\[
	\preim (a_1, \ldots, a_m) = \bigcap_{i = 1, m} \preim (a_i)
\]
The preimage of multiple image lines or points can be either
an empty set, a point, a line or a plane.

\begin{figure}[h]
	\centering
	\includegraphics[width=\linewidth]{img/todo.png}
	\caption{Preimage and coimage of points and lines. (video 10, minute 34)}%
	\label{fig:preimage_coimage}
\end{figure}

Figure~\ref{fig:preimage_coimage} shows the preimage and coimage of a point $p$
on a line $L$.
\begin{itemize}
	\item Preimages $P_1$ and $P_2$ of the image lines should intersect on the line $L$.
	\item Preimages of the two image points $\bm{x}_1$ and $\bm{x}_2$
		should intersect in the point $p$.
	\item Normals $l_1$ and $l_2$ define the coimages of the line $L$.
\end{itemize}


\subsubsection*{Preimage and Coimage of Points and Lines}%
\label{ssub:preimage_and_coimage_of_points_and_lines}

For a \textbf{moving camera at time $t$}, let $\bm{x}(t)$ denote the image coordinates
of a 3D point $\bm{X}$ in homogeneous coordinates:
\[
	\lambda(t) \bm{x}(t) = K(t) \Pi_0 g(t) \bm{X}
\]
where $\lambda(t)$ denotes the depth of the point, $K(t)$ the intrinsic parameters,
$\Pi_0$ the generic projection, and
\[
	g(t) = \begin{pmatrix}
		R(t) & T(t) \\
		0 & 1
	\end{pmatrix}
	\in SE(3)
\]
denotes the rigid body motion at time $t$.
Let us consider a \textbf{3D line $L$} in homogeneous coordinates:
\[
	L = \Set {\bm{X}} {\bm{X} = \bm{X}_0 + \mu \bm{V}, \mu \in \R} \subset \R^4
\]
where $\bm{X}_0 = \tr{[ X_0, Y_0, Z_0, 1 ]} \in \R^4$ are the coordinates
of the base point $p_0$ and the vector $\bm{V} = \tr{[V_1, V_2, V_3, 0]} \in \R^4$
is a nonzero vector indicating the line direction.\\

The \textbf{preimage of $L$} with respect to the image at time $t$ is a plane $P$
with normal $l(t)$, where $P = \text{span}(\widehat{l})$.
The vector $l(t)$ is orthogonal to all points $\bm{x}(t)$ of the line:
\[
	\tr{l(t)} \bm{x}(t) = \tr{l(t)} K(t) \Pi_0 g(t) \bm{X} = 0
\]
Assume we are given a \textbf{set of $m$ images at times} $t_1, \ldots, t_m$
where
\[
	\lambda_i = \lambda(t_i),
	\bm{x}_i = \bm{x}(t_i),
	l_i = l(t_i),
	\Pi_i = K(t_i) \Pi_0 g(t_i)
\]
With this notation, we can relate the \textbf{$i$-th image of a point $p$}
to its world coordinates $\bm{X}$:
\[
	\lambda_i \bm{x}_i = \Pi_i \bm{X}
\]
and the \textbf{$i$-th coimage of a line $L$}
to its world coordinates $(\bm{X}_0, \bm{V})$:
\[
	\tr{l_i} \Pi_i \bm{X}_0 = \tr{l_i} \Pi_i \bm{V} = 0
\]


\subsection{From Preimages to Rank Constraints}%
\label{sub:from_preimages_to_rank_constraints}


The above equations contain the \textbf{3D parameters of points and lines}
as unknowns. As in the two-view case, we wish to
\textbf{eliminate these unknowns} so as to obtain relationships between
the 2D projections and the camera parameters.\\

In the two-view case an elimination of the 3D coordinates lead to the
\textbf{epipolar constraint} for the essential matrix $E$ or
(in the uncalibrated case) the fundamental matrix $F$.
The 3D coordinates (depth values $\lambda_i$ associated with each point)
could subsequently be obtained from another contraint.\\

There exist \textbf{different ways to eliminate the 3D parameters}
leading to different kinds of constraints which have been studied in Computer Vision.\\

A systematic elimination of these constraints will lead to a complete set of conditions.


\subsubsection*{Point Features}%
\label{ssub:point_features}

Consider images of a 3D point $\bm{X}$ seen in multiple views:
\[
	\I \lambda \equiv
		\begin{pmatrix}
			\bm{x}_1 & 0 & \cdots & 0 \\
			0 & \bm{x}_2 & \cdots & 0 \\
			\vdots & \vdots & \ddots & \vdots \\
			0 & 0 & \cdots & \bm{x}_m
		\end{pmatrix}
		\begin{pmatrix}\lambda_1 \\ \lambda_2 \\ \vdots \\ \lambda_m \end{pmatrix}
		= \begin{pmatrix}\Pi_1 \\ \Pi_2 \\ \vdots \\ \Pi_m \end{pmatrix} \bm{X}
\]
which is of the form
\[
	\I \lambda = \Pi \bm{X}
\]
where $\lambda \in \R^m$ is the \textbf{depth scale vector},
and $\Pi \in \RR{3m}{4}$ the \textbf{multiple-view projection matrix}
associated with the \textbf{image matrix} $\I \in \RR{3m}{m}$.\\

Note that apart from the 2D coordinates $\I$, everything else in the above
equations is unknown. As in the two-view case, the goal is to decouple
the above equation into constraints which allow to separately recover
the camera displacements $\Pi_i$ on one hand and the scene
structure $\lambda_i$ and $\bm{X}$ on the other hand.\\

Every column of $\I$ lies in a four-dimensional space spanned by colummns
of the matrix $\Pi$. In order to have a solution to the above equation,
the columns of $\I$ and $\Pi$ must therefore be linearly dependent,
in other words, the matrix
\[
	N_p \equiv (\Pi, \I)
	= \begin{pmatrix}
		\Pi_1 & \bm{x}_1 & 0 & \cdots & 0 \\
		\Pi_2 & 0 & \bm{x}_2 & \cdots & 0 \\
		\vdots & \vdots & \vdots & \ddots & \vdots \\
		\Pi_m & 0 & 0 & \cdots & \bm{x}_m
	\end{pmatrix}
\]
where $N_p \in \RR{3m}{(4+m)}$, must have a nontrivial right null space.
For $m \geq 2$ (i.e. $3m \geq m + 4$), full rank would be $m+4$.
Linear dependence of columns therefore implies the \textbf{rank constraint}:
\[
	\boxed{\text{rank}(N_p) \leq m + 3 }
\]
In fact, the vector $u \equiv \tr{( \tr{\bm{x}}, - \tr{\lambda} )} \in \R^{m+4}$
is in the right null space, since $N_p u = 0$.\\

For a more compact formulation of the above rank constraint,
we introduce the matrix
\[
	\I^{\bot} \equiv
		\diagtwo{\widehat{\bm{x}_1}}{\widehat{\bm{x}_m}}
		\in \RR{3m}{3m}
\]
which has the property of ``annihilating'' $\I$:
\[
	\I^{\bot} \I = 0
\]
We can premultiply the above equation to obtain
\[
	\I^{\bot} \Pi \bm{X} = 0
\]
Thus the vector $\bm{X}$ is in the null space of the matrix
\[
	W_p
		\equiv \I^{\bot} \Pi
		= \begin{pmatrix}
			\widehat{\bm{x}_1} \Pi_1 \\ \vdots \\ \widehat{\bm{x}_m} \Pi_m
		\end{pmatrix}
		\in \RR{3m}{4}
\]
To have a nontrivial solution, we must have
\[
	\text{rank}(W_p) \leq 3
\]
If all images $\bm{x}_i$ are from a single 3D point $\bm{x}$,
then the matrix $W_p$ should only have a one-dimensional null space.
Given $m$ images $\bm{x}_i \in \R^3$ of a point $p$ with respect to $m$ camera
frames $\Pi_i$, we must have the rank condition
\[
	\boxed{\text{rank}(W_p) = \text{rank}(N_p) - m \leq 3}
\]


\subsubsection*{Line Features}%
\label{ssub:line_features}

We can derive a similar rank constraint for lines.
As we saw above, for the coimages $l_i, i=1, \ldots, m$ of a line $L$
spanned by a base $\bm{X}_0$ and a direction $\bm{V}$ we have:
\[
	\tr{l_i} \Pi_i \bm{X}_0 = \tr{l_i} \Pi_i \bm{V} = 0
\]
Therefore the matrix
\[
	W_l \equiv \begin{pmatrix}
		\tr{l_1} \Pi_1 \\ \vdots \\ \tr{l_m} \Pi_m
	\end{pmatrix}
	\in \RR{m}{4}
\]
must satisfy the rank constraint
\[
	\boxed{\text{rank}(W_l) \leq 2}
\]
since the null space of $W_l$ contains the two vectors $\bm{X}_0$ and $\bm{V}$.


\subsection{Geometric Interpretation}%
\label{sub:geometric_interpretation}


In the case of a \textbf{point} $\bm{X}$, we had the equation
\[
	W_p \bm{X} = 0, \quad \text{with} \quad
	W_p = \begin{pmatrix}
			\widehat{\bm{x}_1} \Pi_1 \\ \vdots \\ \widehat{\bm{x}_m} \Pi_m
		\end{pmatrix}
		\in \RR{3m}{4}
\]
Since all matrices $\widehat{\bm{x}_i}$ have rank 2, the number of independent
rows in $W_p$ is at most $2m$. These rows define a set of $2m$ planes.
Since $W_p \bm{X} = 0$, the point $\bm{x}$ is in the intersection of all these planes.
In order for the $2m$ planes to have a unique intersection,
we need to have $\text{rank}(W_p) = 3$.\\

In the case of a line $L$ in two views, we have the equation
\[
	\text{rank}(W_l) \leq 2, \quad \text{with} \quad
	W_l = \begin{pmatrix}
		\tr{l_1} \Pi_1 \\ \tr{l_2} \Pi_2
	\end{pmatrix}
	\in \RR{2}{4}
\]
Clearly, we already have $\text{rank}(W_l) \leq 2$, so there is no intrinsic
constraint on two images of a line: The preimage of two image lines
always contains a line. We shall see that this is no longer true for three
or more images of a line, then the above constraint really becomes meaningful.


\subsection{The Multiple-View Matrix}%
\label{sub:the_multiple_view_matrix}


In the following, the rank constraints will be rewritten in a more compact
and transparent manner. Let us assume we have $m$ images, the first of which
is in world coordinates. Then we have projection matrices of the form
\[
	\Pi_1 = [R_1,T_1] = [I,0], \ldots, \Pi_m = [R_m, T_m] \ \in \RR{3}{4}
\]
which model the projection of a point $\bm{X}$ into the individual images.\\

In general for uncalibrated cameras (i.e.\ $K_i \neq I$), $R_i$ will not be
an orthogonal rotation matrix but rather an arbitrary invertible matrix.
Again, we define the matrix $W_p$ as follows:
\[
	W_p \equiv \I^{\bot} \Pi = \begin{pmatrix}
		\widehat{\bm{x}_1} \Pi_1 \\ \vdots \\ \widehat{\bm{x}_m} \Pi_m
	\end{pmatrix}
	\in \RR{3m}{4}
\]
The rank of the matrix $W_p$ is not affected if we multiply by
a full-rank matrix $D_p \in \RR{4}{5}$ as follows:
\begin{align*}
	W_p D_p
		&= \begin{pmatrix}
				\widehat{\bm{x}_1} \Pi_1 \\ \vdots \\ \widehat{\bm{x}_m} \Pi_m
			\end{pmatrix}
			\begin{pmatrix}
				\widehat{\bm{x}_1} & \bm{x}_1 & 0 \\
				0 & 0 & 1
			\end{pmatrix} \\
		&= \begin{pmatrix}
				\widehat{\bm{x}_1} \widehat{\bm{x}_1} & 0 & 0 \\
				\widehat{\bm{x}_2} R_2 \widehat{\bm{x}_1}
					& \widehat{\bm{x}_2} R_2 \bm{x}_1
					& \widehat{\bm{x}_2} T_2 \\
				\vdots & \vdots & \vdots \\
				\widehat{\bm{x}_m} R_m \widehat{\bm{x}_1}
					& \widehat{\bm{x}_m} R_m \bm{x}_1
					& \widehat{\bm{x}_m} T_m
			\end{pmatrix}
\end{align*}
This means that $\text{rank}(W_p) \leq 3$ if and only if the submatrix
\[
	M_p \equiv \begin{pmatrix}
		\widehat{\bm{x}_2} R_2 \bm{x}_1
			& \widehat{\bm{x}_2} T_2 \\
		\vdots & \vdots \\
		\widehat{\bm{x}_m} R_m \bm{x}_1
			& \widehat{\bm{x}_m} T_m
	\end{pmatrix}
	\in \RR{3(m-1)}{2}
\]
has $\text{rank}(M_p) \leq 1$.
The matrix $M_p$ is called the \textbf{multiple-view matrix} associated
with a point $p$. It involves both the image $\bm{x}_1$ in the first view
and the coimages $\widehat{\bm{x}_i}$ in the remaining views.
In summary, \textbf{for multiple images of a point $p$, the matrices}
\textbf{$N_p$, $W_p$ and $M_p$ satisfy}:
\[
	\boxed{ \begin{array}{rl}
		\text{rank}(M_p)
		& = \text{rank}(W_p) - 2 \\
		& = \text{rank}(N_p) - (m + 2) \\
		& \leq 1
	\end{array}}
\]
The constraint $\text{rank}(M_p) \leq 1$ implies that the two columns
are linearly dependent. In fact we have
$\lambda_1 \widehat{\bm{x}_i} R_i \bm{x}_1 + \widehat{\bm{x}_i} T_i = 0,
i = 2, \ldots, m$ which yields
\[
	M_p \begin{pmatrix} \lambda_1 \\ 1 \end{pmatrix} = 0
\]
Therefore the coefficient capturing the linear dependence
is simply the distance $\lambda_1$ of the point $p$ from the first camera center.
In other words, the multiple-view matrix captures exactly the information
about a point $p$ that is missing from a single image, but encoded in multiple images.


\subsection{Relation to Epipolar Constraints}%
\label{sub:relation_to_epipolar_constraints}


For the multiple-view matrix
\[
	M_p \equiv \begin{pmatrix}
		\widehat{\bm{x}_2} R_2 \bm{x}_1
			& \widehat{\bm{x}_2} T_2 \\
		\vdots & \vdots \\
		\widehat{\bm{x}_m} R_m \bm{x}_1
			& \widehat{\bm{x}_m} T_m
	\end{pmatrix}
	\in \RR{3(m-1)}{2}
\]
to have $\text{rank}(M_p) = 1$, it is necessary that the pair of vectors
$\widehat{\bm{x}_i} T_i$ and $\widehat{bm{x}_i} R_i \bm{x}_1$ to be
linearly dependent for all $i = 2, \ldots, m$.
This gives the \textbf{epipolar constraints}
\[
	\tr{\bm{x}_i} \widehat{T_i} R_i \bm{x}_1 = 0
\]
between the first and the $i$-th image.
This can be proved by the fact these vectors are proportional if and only if
the planes spanned by $(\bm{x}_i, T_i)$ and $(\bm{x}_i, R_i \bm{x}_1)$ are the same,
leading to their triple product being $0$, which is the epipolar constraint.\\

We shall see that the multiview constraint provides more information
than the pairwise epipolar constraints.


\subsubsection*{Analysis of the Multiple-view Constraint}%
\label{ssub:analysis_of_the_multiple_view_constraint}

For any nonzero vectors $a_i, b_i \in \R^3, i = 1, \ldots, n$, the matrix
\[
	\begin{pmatrix}
		a_1 & b_1 \\
		\vdots & \vdots \\
		a_n & b_n
	\end{pmatrix}
	\in \RR{3n}{2}
\]
is rank-deficient if and only if $a_i \tr{b_j} - b_i \tr{a_j} = 0$
for all $i,j = 1, \ldots, n$ (not proved here).
Applied to the rank constraint on $M_p$ we get
\[
	\widehat{\bm{x}_i} R_i \bm{x}_1 \tr{(\widehat{\bm{x}_j} T_j)} -
	\widehat{\bm{x}_i} T_i \tr{(\widehat{\bm{x}_j} R_j \bm{x}_1)} = 0
\]
which gives the \textbf{trilinear constraint}
\[
	\widehat{\bm{x}_i}
	(T_i \tr{\bm{x}_1} \tr{R_j} - R_i \bm{x}_1 \tr{T_j})
	\widehat{\bm{x}_j} = 0
\]
This is a matrix equation giving $3 \times 3 = 9$ scalar
trilinear equations, only four of which are linearly independent.\\

Except for degeneracies, the \textbf{bilinear (epipolar) constraints}
relating two views are already contained in the
\textbf{trilinear constraints} obtained for the multiview scenario.\\

Note that the \textbf{equivalence} between the bilinear and trilinear constraints
on one hand and the condition that $\text{rank}(M_p) \leq 1$ on the other
only holds if the vectors in $M_p$ are nonzero.
In certain degenerate cases this is not fulfilled.


\subsubsection*{Uniqueness of the Preimage}%
\label{ssub:uniqueness_of_the_preimage}


Given three vectors $\bm{x}_1, \bm{x}_2, \bm{x}_3 \in \R^3$
and three camera frames with distinct optical centers,
if the three images satisfy the \textbf{pairwise epipolar constraints}
\[
	\tr{\bm{x}_i} \widehat{T_{ij}} R_{ij} \bm{x}_j = 0,
	\quad i,j = 1,2,3
\]
then a unique preimage is determined except if the tree lines
associated to image points $\bm{x}_1, \bm{x}_2, \bm{x}_3$ are coplanar.
Here $T_{ij}$ and $R_{ij}$ refer to the transition between frames $i$ and $j$.
Similarly, if these vectors satisfy all \textbf{trilinear constraints}
\[
	\widehat{\bm{x}}_i
	( T_{ji} \tr{\bm{x}_i} \tr{R_{ki}} - R_{ji} \bm{x}_i \tr{T_{ki}} )
	\widehat{\bm{x}}_k = 0,
	\quad i,j,k = 1,2,3
\]
Then a unique preimage is determined unless the three lines
associated to image points $\bm{x}_1, \bm{x}_2, \bm{x}_3$ are colinear.
(Not proved here).


\subsubsection*{Degeneracies for the Bilinear Constraints}%
\label{ssub:degeneracies_for_the_bilinear_constraints}


\begin{figure}[h]
	\centering
	\includegraphics[width=\linewidth]{img/todo.png}
	\caption{Degeneracies and trifocal plane. (video 11, minute 71)}%
	\label{fig:degeneracies_trifocal_plane}
\end{figure}

In Figure~\ref{fig:degeneracies_trifocal_plane},
the point $p$ lies in the plane spanned by
the three optical centers which is also called the \textbf{trifocal plane}.
In this case, all pairs of lines do intersect, yet it does not imply
a unique 3D point $p$ (a unique preimage).
In practice this degenerate case arises rather seldom.


\begin{figure}[h]
	\centering
	\includegraphics[width=\linewidth]{img/todo.png}
	\caption{Degeneracies of rectilinear motion. (video 11, minute 74)}%
	\label{fig:degeneracies_rectilinear_motion}
\end{figure}

In Figure~\ref{fig:degeneracies_rectilinear_motion},
the optical centers lie on a straight line (\textbf{rectilinear motion}).
Again, all pairs of lines may intersect without there being a unique preimage $p$.
This case is frequent in applications when the camera moves in a straight line
(e.g.\ a car on a highway). Then the epipolar constraints will not
allow a unique reconstruction.\\

Fortunately, \textbf{the trilinear constraint assures a unique preimage}
(unless $p$ is also on the same line with the optical centers).


\subsubsection*{Uniqueness of the Preimage}%
\label{ssub:uniqueness_of_the_preimage}

Using the multiple-view matrix, we obtain a more general and simpler
characterization regarding the uniqueness of the preimage.\\

Given $m$ vectors representing the $m$ images of a point in $m$ views,
they correspond to the same point in the 3D space if the rank of
the multiview matrix $M_p$ relative to any of the camera frames is one.
If the rank is zero, the point determined up to the line on which
all the camera centers must lie. In summary we get:
\[
	\boxed{\begin{array}{rl}
		\text{rank}(M_p) = 2\ \Rightarrow
			& \text{no point correspondence} \\
			& \text{\& empty preimage} \\
		\text{rank}(M_p) = 1\ \Rightarrow
			& \text{point correspondence} \\
			& \text{\& unique preimage} \\
		\text{rank}(M_p) = 0\ \Rightarrow
			& \text{point correspondence} \\
			& \text{\& not unique preimage}
	\end{array}}
\]
With these constraints we could decide which features to match
for establishing point correspondence over multiple frames.


\subsection{Multiple-View Reconstruction Algorithms}%
\label{sub:multiple_view_reconstruction_algorithms}


\subsection{Multiple-View Reconstruction of Lines}%
\label{sub:multiple_view_reconstruction_of_lines}


