\section{Representing a Moving Scene}
\label{sec:moving_scene}


\subsection{The Origins of 3D Reconstruction}
\label{sub:the_origins_of_3d_reconstruction}

The goal to reconstruct the three-dimensional structure of the world from
a set of two-dimensional views has long history in computer vision.
It is a classical \textbf{ill-posed problem}, because the reconstruction
consistent with a given set of observations/images is typically not unique.
Therefore, one will need to impose additional assumptions.
Mathematically, the study of geometric relations between a 3D scene
and the observed 2D projections is based on two types of transformations, namely:
\begin{itemize}
	\item \textbf{Euclidean motion} or \textbf{rigid body motion}
		representing the motion of the camera from one frame to the next.
	\item \textbf{Perspective projection} to account for the image formation
		process (see pinhole camera, etc).
\end{itemize}

The notion of perspective projection has its roots among the ancient Greeks
(Euclid of Alexandria, \roughly{} 400 B.C.) and the Renaissance period
(Brunelleschi \& Alberti, 1435).
The study of perspective projection lead to the field of
\textbf{projective geometry} (Girard Desargues 1648, Gaspard Monge 18th cent).\\

The first work on the problem of multiple view geometry was that of
\textbf{Erwin Kruppa (1913)} who showed that two views of five points
are sufficient to determine both the relative tansformation
(\textbf{motion}) between the two views and the 3D location (\textbf{structure})
of the points up to finitely many solutions.\\

A linear algorithm to recover structure and motion from two views based
on the epipolar constraint was proposed by \textbf{Longuet-Higgins}
in \textbf{1981}. An entire series of works along these lines was summarized
in several text books (Faugeras 1993, Kanatani 1993,
Maybank 1993, Weng et al. 1993).\\

Extensions to three views were developed by Spetsakis and Aloimonos '87, '90
, and by Shashua '94 and Hartley '95.
Factorization techniques for multiple views and orthogonal projection were
developed by Tomasi and Kanade 1992.\\

The joint estimation of camera motion and 3D location is called
\textbf{structure and motion} or \textbf{visual SLAM}.


\subsection{3D Space \& Rigid Body Motion}
\label{sub:3d_space_rigid_body_motion}


\subsubsection{Three-dimensional Euclidean Space}
\label{ssub:three_dimensional_euclidean_space}

The three-dimensional Euclidean space $\E^3$ consists of all points
$p \in \E^3$ characterized by coordinates
	\[\bm{X} \equiv \tr{(X_1, X_2, X_3)} \in \R^3\]

such that $\E^3$ can be identified with $\R^3$.
That means we talk about points ($\E^3$) and coordinates ($\R^3$)
as if they were the same thing. Given two points $\bm{X}$ and $\bm{Y}$,
one can define a \textbf{bound vector} as
	\[v = \bm{Y} - \bm{X} \in \R^3\]

Considering this vector independent of its base point $\bm{Y}$ makes
it a \textbf{free vector}. The set of free vectors $v \in \R^3$
forms a linear vector space. By identifying $\E^3$ and $\R^3$,
one can endow $\E^3$ with a scalar product, a norm and a metric.
This allows to compute \textbf{distances, curve length, areas or volumes.}
\[\text{For a curve } \gamma : [0,1] \rightarrow \R^3,\quad
	l(\gamma) \equiv \int_{0}^1 | \dot{\gamma}(s) | d\!s\]


\subsubsection{Cross Product \& Skew-symmetric Matrices}
\label{ssub:cross_product_and_skew_symmetric_matrices}

On $\R^3$ one can define a cross product
\[\times : \R^3 \times \R^3 \rightarrow \R^3,\quad u \times v =
	\begin{pmatrix}
		u_2v_3 - u_3v_2 \\
		u_3v_1 - u_1v_3 \\
		u_1v_2 - u_2v_1
	\end{pmatrix} \in \R^3\]

which is a vector \textbf{orthogonal to $u$ and $v$}.
Since $u \times v = -v \times u$, the cross product introduces an \textbf{orientation}.
Fixing $u$ induces a linear mapping $v \mapsto u \times v$ wich
can be represented by the \textbf{skew-symmetric matrix}
\[\widehat{u} = \hatmat{u_1}{u_2}{u_3} \in \RR{3}{3}\]

In turn, every skew symmetric matrix $M = -\tr{M} \in \RR{3}{3}$
can be identified with a vector $u \in \R^3$.
The operator \^{} defines an \textbf{isomorphism} between $\R³$
and the space $so(3)$ of the $3 \times 3$ skew-symmetric matrices.
Its inverse is denoted by $\vee : so(3) \rightarrow \R^3$.


\subsubsection{Rigid-body Motion}
\label{ssub:rigid_body_motion}

A \textbf{rigid-body motion} (or rigid-body transformation)
is a family of maps
\[g_t : \R^3 \rightarrow \R^3;\quad \bm{X} \mapsto g_t(\bm{X}),\quad t \in [0,T]\]

which preserve the norm and cross product of any two vectors:
\begin{itemize}
	\setlength\itemsep{-0.2em}
	\item $\forall v \in \R^3, \quad |g_t(v)| = |v|$
	\item $\forall u,v \in \R^3, \quad g_t(u) \times g_t(v) = g_t(u \times v)$
\end{itemize}

Since norm and scalar product are related by the \textbf{polarization identity}
\[\inner{u}{v} = \frac{1}{4}(|u+v|^2 - |u-v|^2)\]

one can also state that a rigid-body motion is a map which
preserves inner product and cross product.
As a consequence, rigid-body motions also preserves the \textbf{triple product}
\[\forall u, v, w \in \R^3, \quad
	\inner{g_t(u)}{g_t(v) \times g_t(w)} = \inner{u}{v \times w}\]

which means that they are volume-preserving.


\subsubsection{Representation of Rigid-body Motion}
\label{ssub:representation_of_rigid_body_motion}

Let $g_t$ our rigid body motion. We are going to detail what this transformation
is doing to the initial frame of orthonormal oriented vecors
$e_1, e_2, e_3 \in \R^3$.
We note the transformed vectors $r_i = g_t(e_i)$.
Scalar and cross product of these vectors are preserved:
	\[\tr{r_i}r_j = \tr{g_t(e_i)}g_t(e_j) = \tr{e_i}e_j = \delta_{ij}, \quad
	r_1 \times r_2 = r_3\]

The first constraint amounts to the statement that the matrix
$R = (r_1, r_2, r_3)$ is an orthogonal matrix: $\bm{\tr{R}R=R\tr{R}=I}$,
whereas the second property implies that $\bm{\det(R) = +1}$.
In other words: $R$ is an element of the group
$SO(3) = \{R \in \RR{3}{3}\ |\ \tr{R}R=I,\ \det(R) = +1\}$.\\

The motion of the origin can be represented by a \textbf{translation}
$\bm{T \in R^3}$. Thus the rigid body motion $g_t$ can be written as:
	\[g_t(x) = Rx + T\]


\subsubsection{Exponential Coordinates of Rotation}
\label{ssub:exponential_coordinates_of_rotation}

We will now derive a representation of an \textbf{infinitesimal rotation}.
To this end, we consider a family of rotation matrices $R(t)$
which continuously transform a point from its original location
$(R(0) = I)$ to a different one.
	\[\bm{X}_{\text{trans}}(t) = R(t)\bm{X}_{\text{orig}}, \quad
	\text{with } R(t) \in SO(3)\]

Since $\forall t,\ R(t)\tr{R(t)} = I$, we have:
	\[\frac{d}{dt}(R\tr{R}) = \dot{R}\tr{R} + R\tr{\dot{R}}= 0
	\implies \dot{R}\tr{R} = -\tr{( \dot{R}\tr{R} )}\]

Thus, $\dot{R}\tr{R}$ is a \textbf{skew-symmetric matrix}.
As shown in the section about the \^{} operator, this implies that
there exists a vector $w(t) \in \R^3$ such that:
	\[\dot{R}(t)\tr{R}(t) = \widehat{w}(t)
	\Leftrightarrow \bm{ \dot{R}(t) = \widehat{w}(t)R(t)}\]

Since $R(0) = I$, it follows that $\dot{R}(0) = \widehat{w}(0)$.
Therefore, the \textbf{skew-symmetric matrix $\bm{\widehat{w}(0) \in so(3)}$
gives the first order approximation of a rotation:}
	\[R(dt) = R(0) + dR = I + \widehat{w}(0) dt\]


\subsection{The Lie Group $SO(3)$}
\label{sub:the_lie_group_so_3_}


\subsubsection{Lie Group and Lie Algebra}
\label{ssub:lie_group_and_lie_algebra}

The above calculations showed that the effect of any infinitesimal
rotation $R \in SO(3)$ can be approximated by an element from
the space of skew-symmetric matrices
	\[so(3) = \{ \widehat{w}\ |\ w \in \R^3\}\]

The rotation group $SO(3)$ is called a \textbf{Lie group}.
The space $so(3)$ is called its \textbf{Lie algebra}.\\

\underline{Definition:}
A \textbf{Lie group} (or infinitesimal group) is a smooth manifold that
is also a group, such that the group operations multiplication
and inversion are smooth maps.\\

As shown above: \textbf{The Lie algebra $\bm{so(3)}$ is the tangent space
at the identity of the rotation group $\bm{SO(3)}$.}\\

An \textbf{algebra over a field $\bm{K}$} is a vector space $V$ over $K$
with multiplication on the space $V$.
Elements $\widehat{w}$ and $\widehat{v}$ of the Lie algebra
generally do not commute.
One can define the \textbf{Lie bracket}
\[[\cdot,\cdot]: so(3) \times so(3) \rightarrow so(3);\quad
[\widehat{w},\widehat{v}] \equiv \widehat{w}\widehat{v} - \widehat{v}\widehat{w}\]


\subsubsection{Sophus Lie (1841--1899)}
\label{ssub:sophus_lie_1841_1899_}

\begin{figure}[ht]
\centering
\includegraphics[width=10em]{img/sophus_lie.jpg}
\caption*{Portrait of Marius Sophus Lie}
\end{figure}

Marius Sophus Lie was a Norwegian-born mathematician.
He created the theory of \textbf{continuous symmetry}, and applied it to
the study of geometry and differential equations. Among his greatest
achievements was the discovery that continuous transformation
groups are better understood in their linearized versions
(``Theory of transformation groups'' 1893).
These \textbf{infinitesimal generators} form a structure which is today
known as a \textbf{Lie algebra}. The linearized version of the group law
corresponds to an operation on the Lie algebra known as
the \textbf{commutator bracket} or \textbf{Lie bracket}.
1882 Professor in Christiania (Oslo),
1886 Leipzig (succeeding Felix Klein),
1898 Christiania.


\subsubsection{The Exponential Map}
\label{ssub:the_exponential_map}

Given the infinitesimal formulation of rotation,
we got to the differential equation system:
	\[\left\{ \begin{aligned}
		\dot{R}(t) &= \widehat{w}(t)R(t) \\
		R(0) &= I \\
	\end{aligned}\right.\]

If we assume that $\widehat{w}(t)$ is constant in time ($=\widehat{w}$),
this known equation has the solution:
	\[R(t) = e^{\widehat{w}t}
		= \sum_{n=0}^{\infty} \frac{{(\widehat{w}t)}^n }{n!}
		= I + \widehat{w}t + \frac{{(\widehat{w}t)}^2 }{2!} + \ldots \]

which is a rotation around the axis $w \in \R^3$
by an angle of t (if $\|w\| = 1$). Alternatively, one can absorb
the scalar $t \in \R$ into the skew  symmetric matrix $\widehat{w}$
to obtain $R(t) = e^{\widehat{v}}$ with $\widehat{v} = \widehat{w}t$.
This \textbf{matrix exponential} therefore defines a map from
the Lie algebra to the Lie group:
	\[\exp : so(3) \rightarrow SO(3);\quad \widehat{w}\mapsto e^{\widehat{w}}\]


\subsubsection{The Logarithm of $SO(3)$}
\label{ssub:the_logarithm_of_so_3_}

There is conversely a mapping from the Lie group to the Lie algebra.
For any rotation matrix $R \in SO(3)$, there exists a $w \in \R^3$
such that $R = \exp(\widehat{w})$. Such an element is denoted by
$\widehat{w} = \log(R)$. If $\R \ne I$, we note $r_{ij}$ its coefficients
and $w$ is given by:
	\[\left\{ \begin{aligned}
		|w| &= \inv{\cos}\left(\frac{\text{trace}(R)-1}{2}\right)\\
		\frac{w}{|w|} &= \frac{1}{2\sin(|w|)}
			\begin{pmatrix}
				r_{32} - r_{23} \\
				r_{13} - r_{31} \\
				r_{21} - r_{12} \\
			\end{pmatrix}
	\end{aligned}\right.\]

For $R = I$, we have $|w| = 0$, i.e.\ a rotation by an angle 0.
The above statement says:
\textbf{Any orthogonal transformation $\bm{\R \in SO(3)}$ can be realized
by rotating by an angle $\bm{|w|}$ around an axis
$\bm{\frac{w}{|w|}}$ as defined above}.\\

Obviously the above representation is not unique since for example,
increasing the angle by multiples of $2\pi$ will give the same rotation.


\subsubsection{Rodrigues' Formula}
\label{ssub:rodrigues_formula}

In analogy to the well-known Euler equation
	\[\forall \phi \in \R, \quad  e^{i\phi} = \cos(\phi) + i\ \sin(\phi)\]

we have an expression for skew symmetric matrices $\widehat{w} \in so(3)$:
	\[\boxed{
	e^{\widehat{w}} = I + \frac{\widehat{w}}{|w|} \sin(|w|)
		+ \frac{\widehat{w}^2}{|w|^2} (1 - \cos(|w|))}\]

This is known as \textbf{Rodrigues' formula}.\\

\underline{Proof sketch:}
Let $t = |w|$ and $v = w/|w|$ such that $w = vt$. Then one can show that:
	\[\widehat{v}^2 = v\tr{v} - I \quad
	\text{and}\quad \widehat{v}^3 = -\widehat{v}\]
Thus, by developing the exponential, we get:
	\[e^{\widehat{v}t} = I +
		\underbrace{\left( t - \frac{t^3}{3!} + \cdots \right)}_{\sin(t)}\widehat{v}
	+ \underbrace{\left(\frac{t^2}{2!}-\frac{t^4}{4!}+\cdots \right)}_{1-\cos(t)}
		\widehat{v}^2\]


\subsection{The Lie Group $SE(3)$}
\label{sub:the_lie_group_se_3_}


\subsubsection{Representation of Rigid-body Motions $SE(3)$}
\label{ssub:representation_of_rigid_body_motions_se_3_}

We have seen that the space of rigid-body motions is given by
the group of special Euclidean transformations:
	\[SE(3) \equiv \{ g = (R,T)\ |\ R \in SO(3),\ T \in \R^3\}\]
In homogeneous coordinates:
	\[\boxed{SE(3) \equiv
	\left\{ g = \begin{pmatrix}
		R & T \\
		0 & 1 \\
	\end{pmatrix}\ \middle|\ R \in SO(3), T \in \R^3\right\}}\]

In the context of rigid motions, one can see the difference
between points in $\E^3$ (which can be rotated and translated)
and vectors in $\R^3$ (which can only be rotated).


\subsubsection{The Lie Algebra of Twists}
\label{ssub:the_lie_algebra_of_twists}

Given a continuous family of rigid-body transformations:
	\[g : \R \rightarrow SE(3);\quad g(t) = \begin{pmatrix}
		R(t) & T(t) \\
		0 & 1 \\
	\end{pmatrix}\ \in \RR{4}{4}\]

we consider:
	\[\dot{g}(t)\inv{g}(t) = \begin{pmatrix}
		\dot{R}\tr{R} & \dot{T} - \dot{R}\tr{R}T \\
		0 & 0 \\
	\end{pmatrix}\ \in \RR{4}{4}\]

As in the case of $SO(3)$ the $\dot{R}\tr{R}$ corresponds
to some skew-symmetric matrix $\widehat{w} \in so(3)$. Defining a vector
$v(t) = \dot{T}(t) - \widehat{w}(t)T(t)$, we have:
	\[\dot{g}(t)\inv{g}(t) = \begin{pmatrix}
		\widehat{w}(t) & v(t) \\
		0 & 0 \\
	\end{pmatrix} \equiv \widehat{\xi}(t) \in \RR{4}{4}\]

Multiplying with $g(t)$ from the right, we obtain:
	\[\dot{g} = \dot{g}\inv{g}g = \widehat{\xi}g\]

The $4 \times 4$ matrix $\widehat{\xi}$ can be viewed as a tangent vector
along the curve $g(t)$. $\widehat{\xi}$ is called a \textbf{twist}.
As in the case of $so(3)$, the set of all twists forms the tangent
space which is the \textbf{Lie algebra}
	\[\boxed{se(3) \equiv \left\{ \widehat{\xi} = \begin{pmatrix}
		\widehat{w} & v \\
		0 & 0 \\
	\end{pmatrix}\ \middle|
	\ \widehat{w} \in so(3),\ v \in \R^3 \right\}}\]

	to the \textbf{Lie group $\bm{SE(3)}$}.\\

As before, we can define operators $\wedge$ and $\vee$ to convert between
a \textbf{twist $\bm{\widehat{\xi} \in se(3)}$} and its
\textbf{twist coordinates} $\bm{ \xi \in \R^6 }$:
	\[\widehat{\xi} \equiv \begin{pmatrix} v \\ w \end{pmatrix}^{\wedge}
		\equiv \begin{pmatrix}
			\widehat{w} & v \\
			0 & 0
		\end{pmatrix}\ \in \RR{4}{4}\]

	\[\begin{pmatrix}
		\widehat{w} & v \\
		0 & 0
	\end{pmatrix}^{\vee} \equiv \begin{pmatrix} v \\ w \end{pmatrix} \in \R^6\]

Start of video 4


\subsection{Representing the Camera Motion}
\label{sub:representing_the_camera_motion}


\subsection{Euler Angles}
\label{sub:euler_angles}


